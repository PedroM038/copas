\UseRawInputEncoding
\documentclass[12pt,a4paper]{article}
\usepackage[utf8]{inputenc}
\usepackage[portuguese]{babel}
\usepackage{graphicx}
\usepackage{amsmath}
\usepackage{amsfonts}
\usepackage{amssymb}
\usepackage{listings}
\usepackage{xcolor}
\usepackage{geometry}
\usepackage{hyperref}
\usepackage{fancyhdr}

\geometry{margin=2.5cm}

% Configuração para código Python
\lstset{
    language=Python,
    basicstyle=\ttfamily\small,
    keywordstyle=\color{blue},
    commentstyle=\color{green!60!black},
    stringstyle=\color{red},
    showstringspaces=false,
    breaklines=true,
    frame=single,
    numbers=left,
    numberstyle=\tiny\color{gray}
}

\title{
    \includegraphics[scale=0.8]{logoUFPR.png} \\
    \vspace*{1cm}
    {\huge\bfseries Relatório} \\
    \vspace*{1cm}
    {\Large Implementação Distribuída do Jogo Copas} \\
    \vspace*{7cm}
}
\author{Pedro Henrique Marques de Lima \\ Felipe Gonçalves Pereira}
\date{28 de Julho de 2024}

\begin{document}

\maketitle

\newpage
\tableofcontents
\newpage

\section{Introdução}

Este relatório apresenta a implementação de uma versão distribuída do jogo de cartas Hearts, desenvolvida em Python utilizando comunicação UDP. O projeto foi estruturado seguindo princípios de arquitetura modular, separando claramente as responsabilidades entre lógica de jogo, comunicação de rede e protocolo de mensagens.

O Hearts é um jogo de cartas para 4 jogadores onde o objetivo é evitar cartas que conferem pontos (copas e dama de espadas), sendo vencedor aquele que acumular menos pontos ao final do jogo.

\section{Visão Geral da Arquitetura}

A arquitetura do sistema foi projetada com três componentes principais:

\begin{itemize}
    \item \textbf{Game Logic} (\texttt{game.py}): Responsável pelas regras do Hearts e estado do jogo
    \item \textbf{Network Manager} (\texttt{network.py}): Gerencia a comunicação UDP entre os nós
    \item \textbf{Protocol Handler} (\texttt{protocol.py}): Define e processa o protocolo de mensagens
\end{itemize}

Esta separação permite manutenibilidade, testabilidade e possibilita futuras extensões do sistema.

\section{Lógica de Rede}

\subsection{Escolha do Protocolo UDP}

A implementação utiliza UDP (User Datagram Protocol) como protocolo de transporte, uma decisão fundamentada nas seguintes considerações:

\subsubsection{Vantagens do UDP}
\begin{itemize}
    \item \textbf{Baixa latência}: Essencial para jogos em tempo real
    \item \textbf{Simplicidade}: Reduz overhead de implementação
    \item \textbf{Broadcast eficiente}: Facilita comunicação um-para-muitos
    \item \textbf{Controle total}: Permite implementar lógica customizada de confiabilidade
\end{itemize}

\subsubsection{Desvantagens do UDP}
\begin{itemize}
    \item \textbf{Não confiável}: Perda de pacotes pode ocorrer
    \item \textbf{Sem garantia de ordem}: Mensagens podem chegar fora de sequência
    \item \textbf{Sem controle de fluxo}: Necessário implementar controle customizado
\end{itemize}

\subsection{Arquitetura de Rede}

O sistema implementa uma topologia em anel lógico onde cada nó conhece todos os outros:

\begin{lstlisting}[caption=Configuração dos Nós]
ports = [base_port, base_port + 1, base_port + 2, base_port + 3]

nodes = [(address, port) for address, port in node_configs]
\end{lstlisting}

\subsection{Gerenciamento de Conexões}

A classe \texttt{NetworkManager} implementa:

\begin{itemize}
    \item \textbf{Socket UDP} para comunicação bidirecional
    \item \textbf{Thread dedicada} para recepção de mensagens
    \item \textbf{Callback system} para processamento assíncrono
    \item \textbf{Broadcast} para mensagens globais
\end{itemize}

\begin{lstlisting}[caption=Estrutura do NetworkManager]
class NetworkManager:
    def __init__(self, node_index, nodes):
        self.sock = socket.socket(socket.AF_INET, socket.SOCK_DGRAM)
        self.sock.bind(nodes[node_index])
        self.receive_thread = threading.Thread(target=self._receive_messages)
        
    def send_to_all(self, message):
        for i in range(self.total_nodes):
            self.send_message(message, i)
\end{lstlisting}

\section{Protocolo de Comunicação}

\subsection{Design do Protocolo}

O protocolo foi desenvolvido com base em mensagens JSON estruturadas, oferecendo:

\begin{itemize}
    \item \textbf{Flexibilidade}: Fácil extensão para novos tipos de mensagem
    \item \textbf{Legibilidade}: Debugging simplificado
    \item \textbf{Estruturação}: Campos bem definidos e tipados
\end{itemize}

\subsection{Tipos de Mensagem}

O protocolo define os seguintes tipos de mensagem:

\subsubsection{Mensagens de Controle}
\begin{itemize}
    \item \texttt{CONNECT}: Anúncio de conexão de jogador
    \item \texttt{START\_GAME}: Distribuição inicial de cartas
    \item \texttt{TOKEN}: Controle de turno
\end{itemize}

\subsubsection{Mensagens de Jogo}
\begin{itemize}
    \item \texttt{GAME}: Jogadas de cartas
    \item \texttt{END\_TRICK}: Finalização de rodadas
    \item \texttt{SCORES}: Atualização de pontuações
    \item \texttt{NEW\_HAND}: Início de nova mão
    \item \texttt{GAME\_END}: Finalização do jogo
\end{itemize}

\subsection{Estrutura das Mensagens}

\begin{lstlisting}[caption=Exemplo de Mensagem de Jogada]
{
    "type": "GAME",
    "action": "PLAY",
    "card": "A Copas",
    "player": 1
}
\end{lstlisting}

\begin{lstlisting}[caption=Exemplo de Mensagem de Fim de Rodada]
{
    "type": "END_TRICK",
    "winner": 2,
    "points": 4,
    "scores": [0, 2, 4, 1],
    "trick": 5
}
\end{lstlisting}

\section{Configuração da Rede}

\subsection{Configuração Dinâmica}

O sistema suporta configuração flexível através de argumentos de linha de comando:

\begin{lstlisting}[caption=Exemplo de Execução]
python game.py 0 localhost 5000    # Jogador 0 (host)
python game.py 1 192.168.1.100     # Jogador 1 remoto
python game.py 2 localhost         # Jogador 2 local
python game.py 3 localhost         # Jogador 3 local
\end{lstlisting}

\subsection{Descoberta e Conexão}

O processo de inicialização segue o padrão:

\begin{enumerate}
    \item \textbf{Host (Player 0)} inicia e aguarda conexões
    \item \textbf{Clients} se conectam enviando mensagem \texttt{CONNECT}
    \item \textbf{Host} monitora até 4 jogadores conectados
    \item \textbf{Jogo inicia} automaticamente quando todos conectados
\end{enumerate}

\section{Controle de Estado e Sincronização}

\subsection{Algoritmo de Token Ring}

Para garantir ordem nas jogadas, implementamos um algoritmo de token ring:

\begin{itemize}
    \item \textbf{Token exclusivo}: Apenas quem possui o token pode jogar
    \item \textbf{Passagem sequencial}: Token passa na ordem dos jogadores
    \item \textbf{Vencedor de rodada}: Recebe token para próxima rodada
\end{itemize}

\subsection{Consistência de Estado}

A consistência é mantida através de:

\begin{itemize}
    \item \textbf{Broadcast de jogadas}: Todas as ações são compartilhadas
    \item \textbf{Validação local}: Cada nó valida independentemente
    \item \textbf{Sincronização por eventos}: Estado atualizado por mensagens
\end{itemize}

\section{Análise de Vantagens e Desvantagens}

\subsection{Vantagens da Implementação}

\subsubsection{Arquitetura}
\begin{itemize}
    \item \textbf{Modularidade}: Separação clara de responsabilidades
    \item \textbf{Extensibilidade}: Fácil adição de novos recursos
    \item \textbf{Manutenibilidade}: Código organizado e documentado
\end{itemize}

\subsubsection{Rede}
\begin{itemize}
    \item \textbf{Performance}: Baixa latência com UDP
    \item \textbf{Simplicidade}: Implementação direta sem overhead
    \item \textbf{Flexibilidade}: Suporte a diferentes topologias de rede
\end{itemize}

\subsubsection{Protocolo}
\begin{itemize}
    \item \textbf{Clareza}: Mensagens JSON legíveis
    \item \textbf{Debugabilidade}: Fácil análise de tráfico
    \item \textbf{Extensibilidade}: Novos tipos facilmente adicionados
\end{itemize}

\subsection{Desvantagens e Limitações}

\subsubsection{Confiabilidade}
\begin{itemize}
    \item \textbf{Perda de pacotes}: UDP não garante entrega
    \item \textbf{Ordem de mensagens}: Possível chegada fora de sequência
    \item \textbf{Detecção de falhas}: Limitada sem heartbeat
\end{itemize}

\subsubsection{Escalabilidade}
\begin{itemize}
    \item \textbf{Número fixo}: Limitado a 4 jogadores
    \item \textbf{Broadcast}: Tráfego cresce quadraticamente
    \item \textbf{Coordenação}: Dependência do host
\end{itemize}

\subsubsection{Segurança}
\begin{itemize}
    \item \textbf{Sem autenticação}: Jogadores não são verificados
    \item \textbf{Dados não criptografados}: Mensagens em texto claro
    \item \textbf{Vulnerável a ataques}: Flooding, spoofing
\end{itemize}

\section{Possíveis Melhorias}

\subsection{Confiabilidade}
\begin{itemize}
    \item Implementar ACK/NACK para mensagens críticas
    \item Sistema de heartbeat para detecção de falhas
    \item Timeout e retransmissão automática
\end{itemize}

\subsection{Segurança}
\begin{itemize}
    \item Autenticação de jogadores
    \item Criptografia de mensagens
    \item Validação de integridade
\end{itemize}

\subsection{Performance}
\begin{itemize}
    \item Compressão de mensagens
    \item Batching de múltiplas operações
    \item Otimização do protocolo JSON
\end{itemize}

\section{Conclusão}

A implementação apresentada demonstra uma arquitetura sólida para jogos distribuídos, aproveitando as vantagens do UDP para comunicação eficiente. A separação modular entre rede, protocolo e lógica de jogo facilita manutenção e extensões futuras.

Embora existam limitações relacionadas à confiabilidade e segurança, típicas de implementações UDP, o sistema atende adequadamente aos requisitos de um jogo Hearts distribuído, proporcionando uma base sólida para desenvolvimentos futuros.

A escolha de UDP, apesar das desvantagens, mostrou-se apropriada para este contexto, onde a baixa latência e simplicidade de implementação superam os desafios de confiabilidade em um ambiente controlado.

\end{document}